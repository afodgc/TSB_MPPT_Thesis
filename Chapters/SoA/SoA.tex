\fancychapter{State-of-the-Art}
\label{sec:introduction:background}

In this chapter, the main State-of-the-Art concepts regarding the engineering behind an \acs{mppt} charger and solar panels are explained. More specifically, it addresses key challenges such as: what is a solar panel, \acs{mppt} algorithms, \acs{dc}-\acs{dc} converter types, and battery charging techniques. 


\section{Solar Panels}

    \ac{pv} panels are devices that convert solar irradiance into electrical energy. A \acs{pv} panel consists of multiple solar cells electrically connected in series and/or parallel to achieve the desired voltage and current levels. As in conventional electrical systems, the interconnection topology directly affects the electrical characteristics of the panel: series connections increase the output voltage, while parallel connections increase the output current~\cite{bendib_survey_mppt_methods}.

    Series-connected \acs{pv} panels operate at higher voltage levels, which reduces conduction losses in interconnecting cables and improves the efficiency of the \acs{dc}-\acs{dc} conversion stage. 
    Moreover, power converters designed for high voltage and low current operation are generally less complex and more cost-effective than those operating at low voltage and high current. 
    However, series configurations are more sensitive to partial shading or panel failure, as the output current of the entire string is limited by the weakest module~\cite{Shading_Effect_PV_Modules_2011}. 
    In contrast, parallel configurations provide improved tolerance to shading and faults but result in higher current levels and increased conduction losses.
    In this project, series/parallel configuration of the solar panels will not be changed, since it is already pre-defined by \ac{tsb}, but it is important to understand the implications of each configuration. 

    Each cell is made of semiconductor materials, usually silicon. This semiconductor is doped with phosphorus, a group V element, to create a negative type layer. On the other side, a layer is doped with boron, a group III element, to create a positive type layer. This creates a p-n junction, which is essential for the photovoltaic effect~\cite{photovoltaics_System_Design}.

    \begin{figure} [!h]
        \centering
        \includegraphics[width=0.60 \textwidth]{Images/Background/solar_cell.pdf}
        \qquad
        \caption{Principle of operation of a solar panel~\cite{imagem_solar_panel}.}
        \label{fig:Solar_Cell}
    \end{figure}

    When sunlight arrives at the solar cell, the photons energy is absorbed by the semiconductor material. This energy excites electrons, allowing them to escape their atomic bonds and create electron hole pairs. The electric field at the p-n junction drives these free electrons towards the n-type layer and holes towards the p-type layer, generating a flow of electric current when the cell is connected to an external circuit~\cite{photovoltaics_System_Design}, Figure~\ref{fig:Solar_Cell}.

    The energy generated results in voltage and current outputs that are not constant. Voltage and current are interdependent, therefore, a variation in one parameter leads to a corresponding variation in the other. This relationship is nonlinear and can be characterized by a current-voltage (I-V) curve. Furthermore, the power output also varies with operating conditions and can be represented by a power-voltage (P-V) curve. Both curves are represented in Figure~\ref{fig:i-v,p-v}.
    In both curves, there is a specific point where the power output is maximized, known as the \ac{mpp}. This point corresponds to a unique combination of voltage and current, denoted as $V_{mpp}$ and $I_{mpp}$, respectively. Operating the solar panel at this point ensures optimal energy extraction under a given condition.

    \begin{figure} [!h]
        \centering
        \includegraphics[width=0.50 \textwidth]{Images/Background/i-v,p-v.pdf}
        \qquad
        \caption{I-V and P-V curves of a solar panel~\cite{photovoltaics_System_Design}.}
        \label{fig:i-v,p-v}
    \end{figure}


    The characteristic curves of a solar panel depend on various factors, including the quality of the materials used, the solar cells, and environmental conditions such as temperature, solar irradiance, wind, and shades. It is noted that under partial
shading conditions, it is possible to have multiple local maxima, but overall, there is still only one true MPP~\cite{Comparison_PV_Array_MPPT_Techniques_2007}.

    As seen in Figure~\ref{fig:irradiance_i-v}, the variation of the solar irradiance changes the I-V and P-V curves of the solar panel. The increase in irradiance produces more power by mainly increasing the current output of the panel. On the other hand, the temperature has an opposite effect (Figure~\ref{fig:temperature_i-v}). The increase in temperature produces a decrease in power output, mainly by decreasing the voltage output of the panel. 

    \begin{figure}[!h]
        \centering
        \begin{subfigure}[b]{0.48\textwidth}
            \centering
            \includegraphics[width=1 \textwidth]{Images/Background/irradiance_i-v.pdf}
            \qquad
            \caption{Cell temperature of 25~ºC and variable Irradiance. }
            \label{fig:irradiance_i-v}
        \end{subfigure}
        \hfill
        \begin{subfigure}[b]{0.48\textwidth}
             \centering
            \includegraphics[width=1 \textwidth]{Images/Background/temperature_i-v.pdf}
            \qquad
            \caption{Irradiance of 1~$\mathrm{kW/m}^2$ and variable temperature.}
            \label{fig:temperature_i-v}
        \end{subfigure}
        \caption{I-V curves under different conditions~\cite{photovoltaics_System_Design}.}
        \label{fig:iv-pv-side-by-side}
    \end{figure}
    

    Both environmental conditions have a meaningful impact on the \acs{mpp} of the solar panel and therefore on the power output of the system. To maximize the power extracted by the solar panels, a classic power converter, like a buck converter with fixed duty cycle, will not be efficient since neither the current nor the voltage are constant for different conditions. Instead, an \acs{mppt} charger is used, which will find the \acs{mpp} and operate the solar panel at that point while sensing the changes caused by the environment.

\section{Maximum Power Point Tracker (MPPT)}
    A \acf{mppt} charger is a device that is used to optimize the power output from solar panels by continuously tracking and adjusting the operation point of the panels to ensure they operate at their \ac{mpp}. 

    To achieve its goal, an \ac{mppt} charger is usually composed of a \acs{dc}-\acs{dc} converter, a microcontroller, and some sensors (Figure~\ref{fig:block diagram mppt}). The sensors are used to measure the input/output variables of the control system (typically voltage and current). The information acquired by these sensors is processed by the microcontroller, which runs an algorithm to determine if the \ac{mpp} was reached or how to act on the system towards the \ac{mpp}. In the second case, the microcontroller generates a signal to control the \acs{dc}-\acs{dc} converter that will adjust its operation accordingly.
 
    \begin{figure} [!h]
        \centering
        \includegraphics[width=0.65 \textwidth]{Images/Background/Block diagram.drawio.pdf}
        \qquad
        \caption{Simplified structure of an \acs{mppt} charger.}
        \label{fig:block diagram mppt}
    \end{figure}
    
    The output can be connected to different types of loads, like batteries, \acs{dc} loads or inverters. This project focuses on batter charging applications.

\section{MPPT Algorithms}
    There are plenty of \ac{mppt} algorithms, each one with its own advantages and disadvantages. The choice of the algorithm will depend on the specific application, the desired performance, and the available resources. 
    In this work, the aim is to extract the maximum energy possible of a solar panel installed in a moving boat, which will produce an I-V curve variation due to the quick changes in irradiance caused by changes of inclination, orientation and clouds, and also temperature changes due to waves and wind.
    For these reasons, tracking speed and accuracy are the most important factors to consider.

    With this in mind, the most relevant algorithms are briefly reviewed in order to identify the most suitable option for this project.

    \subsection{Constant Voltage (CV)}
        This is the simplest and most inefficient method. It works by operating the solar panel at a fixed voltage. This referenced voltage is used to calculate the duty cycle of the \acs{dc}-\acs{dc} converter that will maintain the operating point of the solar panel.
        

         Well-known variants improve performance by setting the reference voltage to a fraction of the open-circuit voltage (typically 71-78\%)~\cite{Comparison_PV_Array_MPPT_Techniques_2007}. This voltage is obtained by briefly disconnecting the panel from the load, increasing reference accuracy and efficiency. Another variant uses a fraction of the short-circuit current as the reference~\cite{MPPT_Comparison_techniques}.
         

        Although both methods are simple and cost-effective, they are inherently inefficient due to the necessity of interrupting energy production during measurement phases. However, alternative implementations address this limitation through proxy measurement techniques employing dummy cells or diodes, whose physical properties closely approximate those of standard solar cells, thereby enabling continuous power generation during measurement acquisition~\cite{Comparison_PV_Array_MPPT_Techniques_2007}~\cite{Optimum_Operating_Point_Tracker_2004}. Still, the \ac{mpp} is not always located at the same percentage of the open-circuit voltage or short circuit current, leading to a lack of accuracy and therefore lower efficiency~\cite{Comparative_study_MPPT_algorithms_2000}.


    \subsection{Perturb and Observe (P\&O)}
        The \ac{peo} method is widely used in commercial products and is the basis of many advanced algorithms. Its popularity lies in its simplicity, low cost, and ease of implementation.

        As its name indicates, the algorithm perturbs the voltage of a \acs{pv} array and observes the resulting effect on the output power~\cite{chermitti2012improvement}. If an increase in voltage leads to an increase in output power, the operating point is moving toward the \ac{mpp}, and the algorithm continues to increase the voltage. Conversely, if the output power decreases, the operating point is moving away from the \ac{mpp}, and the direction of the perturbation is reversed~\cite{PeO_and_Newton_Raphson_Comparison,Look-Up-Table_VS_PeO}. Figure~\ref{fig:PeO_flowchart} illustrates the flowchart of the algorithm.
                

        The biggest drawback of this algorithm is that it oscillates around the \ac{mpp}, which produces power losses. This oscillation can be reduced by decreasing the step size of the perturbation, but this will also reduce the tracking speed of the algorithm. This results in an inherent trade-off between tracking speed and accuracy~\cite{MPPT_Comparison_techniques,Comparison_PV_Array_MPPT_Techniques_2007}.

        To mitigate this issue, an adaptive step size can be used, where the step size is larger when the \ac{mpp} is far away and smaller when it is closer. Consequently, the tracking speed is maximized while minimizing the oscillations around the \ac{mpp}. This can be achieved by measuring the power and voltage and calculating the P-V curve slope ($\Delta P / \Delta V$). At the \ac{mpp}, this slope approaches zero, allowing the step size to be reduced accordingly. This adaptive variant is commonly referred to as the differential power perturb-and-observe(dP-\ac{peo})".

        Another weal-know issue of \ac{peo} is that it can get confused in rapidly changing environmental conditions, like fast irradiance changes caused by clouds. In this case, the algorithm can misinterpret the power change caused by the environmental variation as a result of its own perturbation, leading it to move away from the \ac{mpp} instead of towards it~\cite{Comparison_PV_Array_MPPT_Techniques_2007}. For example, if the perturbation was in the wrong direction, but the irradiance increased, the power output would increase, and the algorithm would continue perturbing in the wrong direction (Figure~\ref{fig:PeO_Divergence}).

        \begin{figure}[!h]
            \centering
            \begin{minipage}[b]{0.49\textwidth}
                \centering
                \includegraphics[width=\textwidth]{Images/Background/PeO.pdf}
                \caption{Flow chart of the classic version of Perturb and Observe algorithm~\cite{Look-Up-Table_VS_PeO}.}
                \label{fig:PeO_flowchart}
            \end{minipage}
            \hfill
            \begin{minipage}[b]{0.49\textwidth}
                \centering
                \includegraphics[width=\textwidth]{Images/Background/fast_change.pdf}
                \caption{Divergence of \ac{peo} from MPP as shown in~\cite{4112312}}
                \label{fig:PeO_Divergence}
            \end{minipage}
        \end{figure}

        To address this limitation, an additional condition is incorporated into the algorithm that evaluates two consecutive measurements of $\Delta P$ and $\Delta V$. Specifically, when the signs of consecutive $\Delta P$ measurements do not follow the signs of $\Delta V$, this indicates that the observed power variation results from environmental disturbances rather than the algorithm's perturbation. Consequently, the voltage adjustment is suppressed. This refined variant is designated the two-point algorithm or improved \ac{peo} method~\cite{bendib_survey_mppt_methods}.

        There are other variations of \ac{peo} algorithm that improve its performance, like Variable Step Size \ac{peo} (VSS-\ac{peo}), the three-point \ac{peo}, a-factor \ac{peo}, and more. Unfortunately, due to the scope of this work, it is not possible to explain all of them.


    \subsection{Incremental Conductance (IncCond)}
        The \ac{incond} algorithm is another widely used \ac{mppt} method due to its accuracy and ability to track the \ac{mpp} under rapidly changing environmental conditions.
        
        This algorithm is based on the fact that at the \ac{mpp}, the derivative of power with respect to voltage is zero. By knowing the output voltage and current of the solar panel, the algorithm can calculate the conductance and the incremental conductance~\cite{Intelligent_PV_Module_2006}. 
        
        The following formulas, Equation~\ref{eq:inc_cond_mpp}, show the equation on which this algorithm is based. 

        \begin{equation}
            \frac{dP}{dV} = \frac{d(V \cdot I)}{dV} = I + V \frac{dI}{dV} \rightarrow \frac{1}{V} \times \frac{dP}{dV} = \frac{I}{V} + \frac{dI}{dV}
            \label{eq:inc_cond_mpp}
        \end{equation}

        At the \ac{mpp} point, the slope of the P-V curve is zero, which means that the negative of the conductance, $-I / V$, is equal to the incremental conductance, $dI / dV$, (Figure~\ref{fig:IncCond_graph}). The algorithm compares these two values to determine if the operating point is at, to the left, or to the right of the \ac{mpp}~\cite{MPPT_Comparison_techniques}\cite{Comparison_PV_Array_MPPT_Techniques_2007}:

        \begin{equation}
            \begin{aligned}
            \frac{dP}{dV}=0 \quad&\Longleftrightarrow\quad \frac{dI}{dV}=-\frac{I}{V}
            \qquad &&\text{at MPP} \\[4pt]
            \frac{dP}{dV}>0 \quad&\Longleftrightarrow\quad \frac{dI}{dV}>-\frac{I}{V}
            \qquad &&\text{left of MPP} \\[4pt]
            \frac{dP}{dV}<0 \quad&\Longleftrightarrow\quad \frac{dI}{dV}<-\frac{I}{V}
            \qquad &&\text{right of MPP}
            \end{aligned}
            \label{eq:inc_cond}
        \end{equation}


        Based on these conditions, the algorithm adjust its operating point (increasing or decreasing the voltage) using a variable step size based on the instantaneous conductance relative to the incremental conductance, Figure~\ref{fig:IncCond_flowchart}. For example, as suggested in~\cite{Comparison_PV_Array_MPPT_Techniques_2007}, the step size can be calculated using a proportional integral (PI) controller with zero as input reference and error defined as:

        \begin{equation}
            e = \frac{I}{V} + \frac{dI}{dV}
        \end{equation}

        Comparative analysis demonstrates that the \ac{incond} algorithm exhibits superior performance relative to the \acs{peo} method. This outcome is expected, as the Incremental Conductance technique was developed specifically to address the limitations inherent in the \acs{peo} approach~\cite{Intelligent_PV_Module_2006}. For example, the \ac{incond} method eliminates the oscillation problem around the MPP that is characteristic of the \acs{peo} method. Additionally, \ac{incond} demonstrates superior performance under rapidly changing environmental conditions, such as sudden irradiance variations, since it uses derivative-based information that is less sensitive to transient disturbances~\cite{bendib_survey_mppt_methods}. 

        \begin{figure}[!h]
            \centering
            \begin{minipage}[b]{0.41\textwidth}
                \centering
                \includegraphics[width=\textwidth]{Images/Background/IncCond_flow.pdf}
                \caption{Flowchart of the \ac{incond} method with direct control~\cite{IncCond_MPPT_Cuk_2011}.}
                \label{fig:IncCond_flowchart}
            \end{minipage}
            \hfill
            \begin{minipage}[b]{0.41\textwidth}
                \centering
                \includegraphics[width=\textwidth]{Images/Background/InCond_graph.pdf}
                \caption{\ac{incond} method principle.}
                \label{fig:IncCond_graph}
            \end{minipage}
        \end{figure}

    
        \subsection{Look Up Table (LUT)}
            The \ac{lut} method is a simple and fast \ac{mppt} algorithm that relies on pre-calculated/measured data to determine the optimal operating point of a solar panel~\cite{MPPT_Comparison_techniques}.
            
            The \acs{lut} table contains several entries organized by voltage and current. Each of these entries contains the optimal duty cycle of the DC-DC converter, that are pre-calculated or measured for a specific voltage and current. This approach allows the algorithm to reach the \acs{mpp} in one clock cycle by measuring the voltage and current of the solar panel, looking for the closest entry in the LUT table, and applying the corresponding duty cycle to the converter. The entries of the table can be calculated through a mathematical model of the solar panel with different conditions (temperature and irradiance) or measured in a laboratory.

            Table~\ref{tab:LUT_matrix} is a visual representation of a 2D \acs{lut}, only based on current and voltage. Each entry represents the duty cycle used in each case.

            The main advantage of this method is its speed since it can reach the \ac{mpp} in one clock cycle. This makes it suitable for applications where fast tracking is required, like in this project. However, it is not very accurate~\cite{Look-Up-Table_VS_PeO}.

            \begin{table}[h]
                \centering
                \caption{\acs{lut} matrix representation with $m \times n$ elements.}
                \label{tab:LUT_matrix}
                \renewcommand{\arraystretch}{1.3}
                \begin{tabular}{c|ccccccc}
                    & $V_1$ & $\cdots$ & $V_k$ & $\cdots$ & $V_m$ \\
                    \hline
                    $I_1$ & $D_{11}$ & $\cdots$ & $D_{1k}$ & $\cdots$ & $D_{1m}$ \\
                    $\vdots$ & $\vdots$ &  & $\vdots$ &  & $\vdots$ \\
                    $I_j$ & $D_{j1}$ & $\cdots$ & $D_{jk}$ & $\cdots$ & $D_{jm}$ \\
                    $\vdots$ & $\vdots$ &  & $\vdots$ &  & $\vdots$ \\
                    $I_n$ & $D_{n1}$ & $\cdots$ & $D_{nk}$ & $\cdots$ & $D_{nm}$
                \end{tabular}
            \end{table}

            To improve the accuracy, the \acs{lut} table needs to be larger, which increases the required memory resources. Also, by adding the temperature and/or irradiance to the table would increase the accuracy, making it almost 100\% accurate, but this would increase the size of the table exponentially. In extreme cases, the memory requirements became expensive and slow. There is a trade-off between size, speed, and accuracy that needs to be considered when using this method. Also, the \acs{lut} method does not adapt to the aging of the solar panel or changes in its characteristics over time, unless the table is updated periodically through new measurements.
            

\section{Types of DC-DC Converters}
    The \acs{dc}-\acs{dc} converter is a crucial component of an \acs{mppt} charger, as it allows the adjustment of the operating point of the solar panel to match the \ac{mpp}. His choice is highly important to the project since it directly affects the efficiency, cost, and complexity of the overall system.

    \acs{dc}-\acs{dc} converters can be classified into two main categories: isolated and non-isolated converters. The main difference between these two types of converters is the presence or absence of galvanic isolation between the input and output circuits~\cite{flexpower_isolated_vs_nonisolated_2025}.


    \subsection{Non-isolated versus isolated DC-DC converters}
        This galvanic isolation is usually achieved by using a transformer, which provides electrical separation between the input and output sides of the converter. This isolation is important in applications where safety is a concern, such as in medical devices or industrial equipment, as it helps to prevent electrical shock and damage to sensitive components. In the context of solar energy systems and E-mobility, isolated converters are often used when the solar panel is connected to the grid~\cite{Review_Isolated_Non_isolated_DC_DC_Converters_MVDC_2021}, as they help to protect against voltage spikes and other electrical disturbances. Also, this converter can provide very higher/lower gain, which is useful for applications where 
        solar panels \ac{mpp} voltage is very different from the battery or load voltage (typically medium/high voltages). 

        In contrast, non-isolated converters do not provide galvanic isolation between the input and output circuits. However, they generally offer higher efficiency, lower cost, reduced size, and lower circuit complexity (Table~\ref{tab:switching-vs-isolated-converters}). Additionally, they present fewer challenges in thermal management and are simpler to design. These advantages, together with the fact that galvanic isolation is not a requirement for this application, led to the exclusion of isolated \acs{dc}-\acs{dc} converters from this project.

        \begin{table}[htbp]
            \centering
            \caption{Comparison between Non-isolated and isolated converters (\fcolorbox{black}{light_red}{undesired}, \fcolorbox{black}{light_yellow}{medium} and \fcolorbox{black}{light_green}{good} characteristics are highlighted).}
            \label{tab:switching-vs-isolated-converters}
            \scriptsize
            \setlength{\tabcolsep}{5pt}
            \renewcommand{\arraystretch}{1.1}
            \begin{adjustbox}{max width=0.95\textwidth}
                \begin{tabularx}{\textwidth}{>{\raggedright\arraybackslash}p{4cm}
                                            >{\centering\arraybackslash}X
                                            >{\centering\arraybackslash}X}
                \toprule
                Type & Non-isolated Converters & Galvanic Isolated Converters \\
                \midrule

                Circuit Complexity & \G Low & \R High \\
                Efficiency & \G High & \Y Medium \\
                Size & \G Compact & \R Large \\
                Electromagnetic Interference (EMI) & \Y Medium/High &  \R High\\
                Switching Frequency & \Y Medium & \Y Medium \\
                Thermal Management & \G Easier & \Y Moderate \\
                Control Complexity & \Y Moderate & \Y Moderate \\
                Gain & \Y Near 1 & \G Very High/low \\
                Cost & \G Low & \R High \\
                \bottomrule
                \end{tabularx}
            \end{adjustbox}
            \end{table}

            


    \subsection{Non-isolated DC-DC converters}
        Modern non-isolated conversion circuits generally use one of three basic topologies: buck, boost, or buck-boost converters. They are "basic" in the sense that only one switching element is needed. A given topology is used to obtain a specific result, such as voltage step-down, voltage step-up, or hybrid mode~\cite{Comparison_PV_Converter_Technologies_1989}. 

            The \textbf{boost converter} is a step-up \acs{dc}-\acs{dc} converter widely employed in \ac{mppt} controllers when the required output voltage is higher than the solar panel open circuit voltage. 
            It uses an inductor to store energy when the transistor is in the ON state and the diode is reverse-biased. When the transistor is on the OFF state, the inductor releases the stored energy to the output through the diode, increasing the output voltage (Figure~\ref{fig:DC-DC_classic_topologys}b)~\cite{Non_Isolated_DC_DC_Converters_MPPT_Controller_2021}.

            This topology has been extensively documented in the literature and is distinguished by its inherent simplicity, cost-effectiveness, and superior efficiency characteristics~\cite{shahira2022_electrical_design_solar_boat}\cite{Look-Up-Table_VS_PeO}\cite{Intelligent_PV_Module_2006}. Depending on the design variation used and constraints, it can achieve conversion efficiencies up to 98\%~\cite{Review_Isolated_Non_isolated_DC_DC_Converter_PV_Application_2018}. 

            A limitation of this topology is that it cannot emulate an impedance lower than the load impedance, which prevents it from reaching values close to the short-circuit current of the \acs{pv} module. Consequently, it is not compatible with all solar panels or \acs{mppt} algorithms~\cite{Review_Isolated_Non_isolated_DC_DC_Converter_PV_Application_2018,A_New_Application_Buck_Boost_Converters_IV_Curve_PV_Modules_2007}.


            The \textbf{buck converter} is a step-down \acs{dc}-\acs{dc} converter also widely used for applications where the solar panel voltage is higher than the battery or load voltage. Similar to the boost converter, it uses an inductor to store energy when the transistor is in the ON state and the diode is reverse-biased. However, in this case, the input is connected to the output when the transistor is in the ON state. When the transistor is in the OFF state, the inductor releases the stored energy to the output through the diode, decreasing the output voltage (Figure~\ref{fig:DC-DC_classic_topologys}a)~\cite{Review_Isolated_Non_isolated_DC_DC_Converter_PV_Application_2018}.

            This converter also has a similar downside to the boost converter. The buck converter can not emulate a smaller impedance than the load impedance, and therefore, it can not reach values near the open circuit voltage of the PV module~\cite{Review_Isolated_Non_isolated_DC_DC_Converter_PV_Application_2018}\cite{A_New_Application_Buck_Boost_Converters_IV_Curve_PV_Modules_2007}.

            \textbf{Buck-Boost converters} are step-up and step-down \acs{dc}-\acs{dc} converters that can operate in both modes. This makes them more versatile than the previous two topologies, but also more complex and less efficient~\cite{Review_Isolated_Non_isolated_DC_DC_Converter_PV_Application_2018}. This converter uses an inductor to store energy when the transistor is in the ON state and the diode is reverse-biased. When the transistor is in the OFF state, the inductor releases the stored energy to the output through the diode, increasing or decreasing the output voltage (Figure~\ref{fig:DC-DC_classic_topologys}c)~\cite{Comparison_PV_Converter_Technologies_1989}.

            \begin{figure} [!h]
                \centering
                \includegraphics[width= \textwidth]{Images/Background/DC-DC_classic_topologys.pdf}
                \qquad
                \caption{Buck (a), Boost (b) and Buck-Boost (c) DC-DC converters.}
                \label{fig:DC-DC_classic_topologys}
            \end{figure}
 
            The conventional buck-boost converter produces an output voltage inverted with respect to the input which creates challenges on sensor measurements, driving circuit and isolation, consequently, it is unsuitable for the battery charging system~\cite{Non_Isolated_DC_DC_Converters_MPPT_Controller_2021}.


            An enhanced variant of the buck-boost converter is the CUK converter, illustrated in Figure~\ref{fig:DC-DC_buck_bost}a. The main characteristic of the CUK converter is that the currents through inductors $L_1$ and $L_2$ are equal, allowing the use of a common magnetic core, which helps reduce current ripple. However, a major drawback of this topology is that the entire load current must pass through the coupling capacitor $C$. In addition, the converter produces an inverted output voltage~\cite{Non_Isolated_DC_DC_Converters_MPPT_Controller_2021}.

            The issue of output voltage polarity inversion can be addressed by using a single-ended primary inductor converter (SEPIC), as shown in Figure~\ref{fig:DC-DC_buck_bost}b. This makes the SEPIC converter suitable for battery charging applications. Nevertheless, its main drawbacks are the high output current and voltage ripple which increases losses and electromagnetic interference (EMI)~\cite{Comparative_Analysis_Non_Isolated_DC_DC_Converters_Solar_PV_2021}. These limitations are mitigated by the ZETA converter, shown in Figure~\ref{fig:DC-DC_buck_bost}c, which provides improved output current characteristics while maintaining the same output voltage polarity as the SEPIC converter~\cite{PeO_and_Newton_Raphson_Comparison}\cite{Non_Isolated_DC_DC_Converters_MPPT_Controller_2021}.

            \begin{figure} [!h]
                \centering
                \includegraphics[width= \textwidth]{Images/Background/DC-DC_buck_bost.pdf}
                \qquad
                \caption{CUK (a), SEPIC (b) and ZETA (c) DC-DC converters.}
                \label{fig:DC-DC_buck_bost}
            \end{figure}

            % Landsman converter is another alternative that provides low input current ripple and non-inverted output voltage while keeping a good efficiency. However, it has high voltage and current output ripple, which is not ideal for battery charging applications~\cite{Comparative_Analysis_Non_Isolated_DC_DC_Converters_Solar_PV_2021}.



    \subsection{Comparison between Non-isolated DC-DC converters}
        All the non-isolated \acs{dc}-\acs{dc} topologies explained before have their own advantages and disadvantages. The choice of the topology will depend on the specific application, the desired performance, and the available resources. To help with this choice, Table~\ref{tab:dc-dc-comparison} shows a comparison between the most relevant characteristics of each topology~\cite{Comparative_Analysis_Non_Isolated_DC_DC_Converters_Solar_PV_2021}. The most significant advantages of each topology are highlighted.

        \begin{table}[htbp]
        \centering
        \caption{Comparison of non-isolated DC-DC topologies (\fcolorbox{black}{light_green}{good} and \fcolorbox{black}{dark_green}{important}  characteristics are highlighted).}
        \label{tab:dc-dc-comparison}
        \scriptsize
        \setlength{\tabcolsep}{4pt}
        \renewcommand{\arraystretch}{1.05}
        \begin{adjustbox}{max width=0.95\textwidth}
            \begin{tabularx}{\textwidth}{>{\raggedright\arraybackslash}p{2.7cm}
                                        >{\centering\arraybackslash}X
                                        >{\centering\arraybackslash}X
                                        >{\centering\arraybackslash}X
                                        >{\centering\arraybackslash}X
                                        >{\centering\arraybackslash}X
                                        >{\centering\arraybackslash}X}
            \toprule
            \textbf{Topology} & \textbf{Boost} & \textbf{Buck} & \textbf{Buck-Boost} & \textbf{ZETA} & \textbf{CUK} & \textbf{SEPIC} \\
            \midrule
            \multicolumn{7}{l}{\textit{\textbf{Electrical performance}}} \\
            Efficiency & \DG High & \DG High & \DG High & \DG Medium/High &  Medium & Medium \\
            EMI & \G Low & \G Low & High & Medium & Medium & Medium \\
            Input current ripple & \G Low  & High & High & \G Low & \G Very low & Medium \\
            Output current ripple & High & \G Low & High & \G Very low & \G Low & Medium \\
            Output voltage ripple & Medium & \DG Low & Medium & \DG Low & Medium & High \\
            \midrule
            \multicolumn{7}{l}{\textit{\textbf{Structural characteristics}}} \\
            Circuit complexity & \G Low & \G Low & \G Low & Medium & Medium & Medium \\
            Size & \G Low & \G Low & \G Low & Medium & Medium & Medium \\
            Storage elements & \G 1 L & \G 1 L & \G 1 L & 2 L, 1 C & 2 L, 1 C & 2 L, 1 C \\
            Switch stress & \G No & Yes & \G No & Yes & Yes & Yes \\
            \midrule
            \multicolumn{7}{l}{\textit{\textbf{Functional aspects}}} \\
            Type & Step-Up & Step-Down & \DG Step-Down/Up & \DG Step-Down/Up & \DG Step-Down/Up & \DG Step-Down/Up \\
            Versatility & Low & Low & Medium & \G High & Medium & \G High \\
            Inverted output & \G No & \G No & Yes & \G No & Yes & \G No \\
            Reliability & \G High & \G High & \G High & Medium & Medium & Medium \\
            \midrule
            \multicolumn{7}{l}{\textit{\textbf{Economic}}} \\
            Cost & \DG Low & \DG Low & \DG Low & Medium & Medium & Medium \\
            \bottomrule
            \end{tabularx}
        \end{adjustbox}
        \end{table}

        For \acs{mppt} controllers, the choice is mainly governed by the possibility of operating at the \ac{mpp} which is connected with the converter type, efficiency, cost (related to the number of storage elements), output voltage ripple.

        The buck converter can be excluded from this comparison since in this project the solar panel voltage is lower than the battery voltage, therefore, a step-down converter is not suitable.
        However, when choosing between the boost and the step-up/down converters, there is a trade-off. 
        The boost converter is simpler, cheaper and more efficient. On the other hand the step-down/up converters are more versatile since they can operate in a wide range of input voltages, which is interesting since the size of the solar panels and battery voltage can change.

        % \textcolor{red}{Option1:} Considering that in this project the solar panel voltage is always lower than the battery voltage, the boost converter is the most suitable topology due to its simplicity, cost-effectiveness, and high efficiency. However, if versatility were to become a priority, the ZETA converter would be the preferred choice among the step-up/down topologies, owing to its non-inverted output voltage and low output current ripple, which are advantageous for battery charging applications.

        % \textcolor{red}{Option2:}
        Considering versatility as a key requirement, the Zeta converter emerges as the most efficient, non-inverted output option. Additionally, its low output voltage ripple makes it particularly well-suited for battery charging applications. Therefore, if versatility is prioritized, the ZETA converter would be the preferred choice among the step-up/down topologies. Also, the buck-boost converter is another alternative that provides versatility, but it has an inverted output voltage that will be a challenge for the sensors and driving circuits. On the other hand, if efficiency were prioritized, the boost converter would be the most suitable topology.




% \section{Switching circuit}
%     \subsection{Types of transistors}
%         \textcolor{red}{falar dos timpos de transistores e como dar drive deles.}
%         \textcolor{red}{BJT, Mosfet, GaN FETs, IGBT, ...}

%     \subsection{Switching problems}
%         \textcolor{red}{diode voltage drop (sync FETs)}

%         \textcolor{red}{Body diode current lekage.}

%         \textcolor{red}{Caso se use um transistor em vez de um diodo tem de se ter cuidado para não provocar curto circuito}

%     \subsection{Protection circuits}
%         \textcolor{red}{Ways to solve the problem}

    
\section{Battery charging techniques}

    As the demand for electronic devices and E-vehicles increases, the need for efficient, compact, and lightweight batteries has emerged. Among many existing technologies, lithium-ion batteries have one of the best energy-to-weight/volume ratios and, at this moment, it is the technology used in \ac{tsb} batteries. But, like every other battery technology, they need to be charged properly to ensure their safety and longevity~\cite{Charging_Algorithms_Lithium_Ion_Batteries_Overview_2012}.

    To charge these batteries properly, the \ac{mppt} algorithm can operate alone if an additional converter is used in series. But in most of \ac{mppt} chargers, the \ac{mppt} and the battery charging unit are integrated in the same converter. This way, the \ac{mppt} algorithm can adjust the operating point of the solar panel to extract the maximum power while the battery charging technique ensures that the battery is charged properly. A proper combination of both algorithms/techniques is essential to ensure both maximum energy extraction and battery safety.

    There are several techniques to charge lithium-ion batteries, the most relevant for this use case being the constant current-constant voltage (\acs{cc}-\acs{cv}) method~\cite{Charging_Algorithms_Nickel_Lithium_Battery_2011}. Also, it is widely used in commercial chargers due to its simplicity and effectiveness~\cite{Review_Different_Charging_Techniques_Lithium_Polymer_Battery_2015}. 
     This method consists of two main phases (Figure~\ref{fig:CC-CV profile}). In the first phase, the battery is charged with a constant current until it reaches a predefined voltage limit (usually 4.2~V per cell). In the second phase, the voltage is held constant at this limit while the current gradually decreases as the battery approaches full charge. Once the current drops below a certain threshold, the charging process is terminated to prevent overcharging~\cite{Review_Different_Charging_Techniques_Lithium_Polymer_Battery_2015}.

     When employing the \acs{cc}-\acs{cv} charging technique with a \ac{mppt} controller, the initial \ac{cc} phase is regulated by the \ac{mppt} algorithm, which extracts the maximum available power from the solar panel limited by the maximum charging current. Once the \ac{cv} phase is reached, the \ac{mppt} function is disabled, and the power converter increases the solar panel voltage to progressively reduce the charging current while keeping the output voltage constant.


    \begin{figure} [!h]
        \centering
        \includegraphics[width=0.45 \textwidth]{Images/Background/CC-CV graph.pdf}
        \qquad
        \caption{Charging profile of CC/CV~\cite{Charging_Algorithms_Lithium_Ion_Batteries_Overview_2012}.}
        \label{fig:CC-CV profile}
    \end{figure}

    More advanced high-performance battery charging techniques have been proposed in literature to reduce charging time without increasing the charging current, thereby preserving battery lifetime. However, these techniques generally introduce higher switching losses and increased system complexity. In the present application, the current generated by the photovoltaic panels is less than half of the maximum allowable battery charging current, consequently, the charging current is inherently limited by the solar panels rather than by the charging strategy. Furthermore, battery lifetime is not a critical concern in this project, as the battery is expected to undergo a limited number of charge cycles, not exceeding 20 cycles per year.

Therefore, the \acs{cc}-\acs{cv} charging method is considered the most suitable technique for this project, owing to its simplicity, effectiveness, and widespread adoption in commercial battery chargers. 

Other charging techniques fall outside the scope of this work, however, some methods are briefly discussed in the references provided~\cite{Charging_Algorithms_Nickel_Lithium_Battery_2011}\cite{Charging_Algorithms_Lithium_Ion_Batteries_Overview_2012}\cite{Review_Different_Charging_Techniques_Lithium_Polymer_Battery_2015}.

    
\section{Processing unit}
    Several types of processing units can be considered for this project, including \acp{mcu}, \acp{fpga}, and \acp{asic}.  

    \acp{asic} are custom-designed integrated circuits optimized for specific applications, offering high efficiency and performance. However, they are typically expensive and time-consuming to design and manufacture. Although commercially available \acp{asic}, commonly referred to as \acp{ic}, mitigate these drawbacks, they are usually made for a specific application and therefore not suitable for the high performance and flexibility required in this project, as discussed in Section~\ref{sec:ICs}.  

        \acp{fpga} are reconfigurable devices capable of implementing custom digital circuits through programmable logic blocks based on look-up tables (LUTs) and interconnections~\cite{231340}. Their main advantage lies in their inherent parallel processing capability, which enables high-performance operation. Nevertheless, \acp{fpga} generally present high costs, high design complexity, and volatile configuration memory, requiring reprogramming after power loss~\cite{ampheo_fpga_vs_microcontroller_2025,FPGA_Implementation_MPPT}.  

        In contrast, \acp{mcu} are widely adopted in embedded systems due to their low cost, ease of use, and versatility. They integrate processing, memory, and peripherals in a single non-volatile device, simplifying hardware design and development. Although their sequential execution limits parallel processing performance compared to \acp{fpga}, modern \acp{mcu} provide sufficient computational capability for many control and signal-processing tasks~\cite{ampheo_fpga_vs_microcontroller_2025}.  

        Considering the emphasis on low cost, reduced development time, and implementation simplicity, a \ac{mcu} is selected for this application. Furthermore, the performance of current \acp{mcu} is adequate to meet the requirements of the \ac{mppt} algorithm and the battery charging strategy.

        A \acf{mcu} typically consists of three fundamental components: a central processing unit (CPU), memory, and peripheral modules. The CPU executes program instructions and performs arithmetic and logical operations. Memory is generally divided into non-volatile memory (Flash) for program storage and volatile memory (SRAM) for data and runtime variables. Peripheral modules provide interfaces to the external environment and may include analog-to-digital converters (ADCs), timers, pulse-width modulation (PWM) units, communication interfaces (such as UART, SPI, I²C and \acs{can}), and general-purpose input/output (GPIO) ports. The integration of these components enables standalone operation without the need for extensive external circuitry~\cite{ibm_microcontroller_2025}.

         When selecting a \acs{mcu} for a specific application, the first factor to consider is to use an 8, 16, or 32 bit \acs{mcu}. For this application, a 16-bit \acs{mcu} would probably be sufficient, however to ensure future-proof and compatibility with other \ac{tsb} systems, the choice is for a 32-bit \ac{mcu}. Besides, nowadays, 32-bit \acp{mcu} are considered the industry standard since most embedded system \ac{red} effort is focused on 32-bit cores, and thus both architectures can achieve similar power consumption~\cite{embedded_upgrading_mcu_designs_2025}.
          There are two main architectures for 32-bit \acp{mcu}: AVR and ARM. Being ARM, the most widely used architecture in the industry and the most advanced, it was chosen for this project~\cite{ampheo_fpga_vs_microcontroller_2025}. 

        Other important characteristics like clock speed, PWM resolution, communication protocols, memory capacity, power consumption, and cost, are not yet well-defined. The choice of \acs{mcu} will be revisited in future stages of the project.

        However, the STM32 family of \acs{mcu} was already selected as it is the standard \acs{mcu} used in \ac{tsb} project and widely used in the industry. The STM32 family offers a wide range of solutions with different characteristics, making it suitable for various applications, including this project. 

        % \textcolor{red}{If there is time, chose a specific STM32 and justify the choice based on some parameters.}

    % \subsection{Sensors}

% \section{Voltage and current sensing}
%     Accurate voltage and current measurements are essential for MPPT operation, since tracking algorithms directly rely on these quantities to estimate power and adjust the operating point of the photovoltaic (PV) array. Measurement noise or inaccuracies directly impact tracking efficiency, especially in high frequency switching DC–DC converters.

%     Voltage sensing is generally straightforward and can be implemented using resistive voltage dividers followed by ADC sampling. Current sensing, however, is more critical and requires careful consideration of the available measurement techniques.



%     \subsection{Overview of current sensing options}

%     Several approaches can be used to measure current in power electronics applications:

%     \begin{itemize}
%         \item Shunt-based sensing with operational amplifiers;
%         \item Dedicated current-sense amplifiers;
%         \item Hall effect current sensor.
%     \end{itemize}

%     Instrumentation amplifiers offer very high accuracy and excellent common-mode rejection, but they are typically more complex, consume more power and are unnecessary for the current and voltage levels involved in this application. Dedicated current-sense amplifiers integrate many features and simplify the design, but they increase cost, reduce flexibility and may impose voltage or interface constraints.

%     In contrast, shunt-based sensing combined with discrete amplifiers provides a good balance between accuracy, simplicity, power consumption and design flexibility, making it well suited for a custom MPPT implementation.

%     \subsection{ Shunt-based sensing with operational and differential amplifiers}

%     The approach to be adopted is based on a low-value shunt resistor placed on the low side of the power path, converting the load current into a small differential voltage. This voltage is then amplified using a differential amplifier built from a precision operational amplifier.

%     Low-side shunt sensing results in a low common-mode voltage and a ground‑referenced output signal, simplifying the analog front‑end and easing direct interfacing with the microcontroller ADC. Compared to high‑side sensing, it also improves noise immunity in a switching environment, which is particularly relevant in DC–DC converters.

%     The differential amplifier ensures that only the voltage drop across the shunt resistor is amplified, rejecting common-mode noise generated by switching activity. Operational amplifiers are especially attractive in this role due to their low power consumption, flexibility in gain selection, price and availability in precision, low‑noise variants suitable for power electronics applications.

%     In a switching MPPT environment, the chosen amplifier must exhibit high common‑mode rejection, low offset voltage and low noise to ensure accurate current measurement. The shunt resistor should have a very low resistance to minimize conduction losses, while maintaining high precision and a low temperature coefficient to reduce measurement error.

\section{Commercial Options}
To assess the feasibility of adopting a \ac{ots} solution, this section reviews existing commercial options for solar battery charging with \ac{mppt} capability. Both specialized \acp{ic} and complete commercial \ac{mppt} chargers are considered, and their electrical, functional, and economic characteristics are evaluated against the requirements of this project. The objective is to identify whether a commercially available solution can satisfy the constraints of the application or, alternatively, to justify the need for a custom-designed converter and control system.

\subsection{Specialized ICs}
\label{sec:ICs}
Several commercially available \acf{ic} are specifically designed for solar battery charging and \ac{mppt} applications. These devices are relevant to this project since, if suitable, they could significantly reduce design complexity by integrating functions such as \acs{dc}-\acs{dc} conversion, protection mechanisms, and \acs{mppt} algorithm.

Table~\ref{tab:commercial-ics-comparison} summarizes four commonly used commercial \acp{ic}. Characteristics marked in red indicate incompatibilities with the project requirements, leading to the exclusion of the corresponding devices.

The BQ24650RVAT is a Stand-Alone Synchronous Buck Battery Charge Controller which implements a high-efficiency battery charger while also providing safety features. By adding a \ac{mcu} to run a \acs{mppt} algorithm and current and voltage sensors, this \acs{ic} can track the \ac{mpp} of a solar panel, making it appropriate for applications with 12/24~V batteries and high-power panels, such as electric vehicles. However, it does not support the required 48~V battery voltage and is therefore unsuitable for this work.

The MAX20801TPBA+ and SPV1040TTR target low-power IoT applications with small batteries and solar panels. Both devices are unable to meet the 48~V output requirement, and their limited publicly available information on tracking speed further restricts performance evaluation.

The LT8491IUKJ\#PBF offers a wide input and output voltage range, buck-boost operation, high efficiency, communication capabilities, and \acs{cc}-\acs{cv} charging. Despite these advantages, its reported \acs{mppt} tracking speed of up to 2~s is insufficient for the dynamic conditions of this application. Additionally, its I\textsuperscript{2}C interface would require an additional microcontroller to provide \acs{can} communication, increasing system cost.

            \begin{table}[htbp]
                \centering
                \caption{Comparison of commercial specialized \acp{ic}
                (\fcolorbox{black}{light_red}{unsuitable} characteristics are highlighted).}
                \label{tab:commercial-ics-comparison}
                \scriptsize
                \setlength{\tabcolsep}{4pt}
                \renewcommand{\arraystretch}{1.05}
                \begin{adjustbox}{max width=0.95\textwidth}
                \begin{tabularx}{\textwidth}{>{\raggedright\arraybackslash}p{2.7cm}
                                            >{\centering\arraybackslash}X
                                            >{\centering\arraybackslash}X
                                            >{\centering\arraybackslash}X
                                            >{\centering\arraybackslash}X}
                \toprule
                \textbf{Model} & \textbf{BQ24650RVAT} & \textbf{MAX20801TPBA+} & \textbf{SPV1040TTR} & \textbf{LT8491IUKJ\#PBF} \\
                \midrule

                \multicolumn{5}{l}{\textit{\textbf{Electrical specifications}}} \\
                Input voltage & \R 5 to 28~V & \R 1.5 to 18~V & \R 0.3 to 5.5~V & 6 to 80~V \\
                Output voltage & \R 12 to 24~V & \R 12.4~V & \R -0.3 to 5.5~V & 1.3 to 80~V \\
                Max output current & Externally limited & 12~A & \R 1.8~A & 10~A \\
                Switching frequency & 600~kHz & Unknown & 100~kHz & 100 to 400~kHz \\
                Electrical efficiency & 95\% & 99.1\% (max) & \R 80 to 95\% & 95 to 99\% \\

                \midrule
                \multicolumn{5}{l}{\textit{\textbf{MPPT and control}}} \\
                Algorithm & User input & Unknown & Unknown & P\&O \\
                Tracking accuracy & User dependent & 99.9\% (max) & \R Unknown & \R Unknown \\
                Tracking speed & User dependent & \R Unknown & \R Unknown & \R 1 to 2~s \\

                \midrule
                \multicolumn{5}{l}{\textit{\textbf{Functional aspects}}} \\
                Topology & Synchronous Buck & Synchronous Buck & Synchronous Boost & Buck-Boost \\
                Communication & \R No & \R No & \R No & \R I\textsuperscript{2}C \\
                Configurability & No & No & No & Yes \\
                Charge profile & CC-CV & \R Unknown & \R Unknown & CC-CV \\
                \midrule
                \multicolumn{5}{l}{\textit{\textbf{Extra hardware needed}}} \\
                Transistors &  Yes & No & No &  Yes \\
                Sensors & Yes & No & No & No \\
                Storage elements & Yes & Yes & Yes & Yes \\

                \midrule
                \multicolumn{5}{l}{\textit{\textbf{Economic}}} \\
                Price & 5.25~€ & 4.50~€ & 3.02~€ & \R 15-20~€ \\

                \bottomrule
                \end{tabularx}
                \end{adjustbox}
            \end{table}


\subsection{Commercial MPPT chargers}
Several commercial \ac{mppt} chargers are available that can address the problem under analysis without requiring additional hardware. However, each solution presents limitations when evaluated against the project requirements. Table~\ref{tab:commercial-mppt-comparison} summarizes selected commercial \acs{mppt} chargers, with red-marked characteristics indicating incompatibilities.

The GVB-8-Li-CV(50.4) is a high-performance and fast \ac{mppt} charger, particularly suited for dynamic environments such as marine applications. It has been successfully used by \acs{tsb} in previous projects, demonstrating high efficiency and reliability. Nevertheless, its lack of configurability, fixed output voltage, and high cost restrict its suitability for this application. Despite these drawbacks, it serves as a valuable reference design.

The Smart Solar MPPT 100/20, developed by Victron Energy, offers high efficiency, fast tracking, and integrated communication interfaces. However, it uses a buck topology that requires higher input voltages, making it incompatible with the small and distributed solar panel arrays used in the \acs{tsb} application.

Other commercial solutions, such as the Rover Lite and MS4840N, provide acceptable performance but rely on communication protocols that are not compatible with the project requirements and offer limited technical documentation.

The Reboost V0.2.1 is an open source \ac{mppt} charger originally developed for automotive applications. While it provides flexibility and near adequate specifications, its reliance on expensive and discontinued components, along with the use of a boost topology, limits scalability and adaptability to potential future battery voltage changes.

Overall, none of the evaluated commercial \ac{mppt} chargers fully satisfy the technical, economic, and configurability requirements of this project, motivating the development of a custom solution.

    \begin{table}[htbp]
        \centering
        \caption{Comparison of commercial MPPT controllers
        (\fcolorbox{black}{light_red}{unsuitable} characteristics are highlighted).}
        \label{tab:commercial-mppt-comparison}
        \scriptsize
        \setlength{\tabcolsep}{3pt}
        \renewcommand{\arraystretch}{1.05}
        \begin{adjustbox}{max width=0.95\textwidth}
        \begin{tabularx}{\textwidth}{>{\raggedright\arraybackslash}p{2.5cm}
                                    >{\centering\arraybackslash}X
                                    >{\centering\arraybackslash}X
                                    >{\centering\arraybackslash}X
                                    >{\centering\arraybackslash}X
                                    >{\centering\arraybackslash}X}
        \toprule
        \textbf{Model} & \textbf{GVB-8-Li-CV (50.4V)} & \textbf{Smart Solar MPPT 100/20} &
        \textbf{Reboost V0.2.1} & \textbf{Rover Lite} & \textbf{MS4840N} \\
        \midrule

        \multicolumn{6}{l}{\textit{\textbf{General information}}} \\
        Brand & Gensun & Victron Energy & TPEE & Renogy & BougeRV \\
        Type & Boost & \R Buck & Synchronous Boost & Boost & Boost \\

        \midrule
        \multicolumn{6}{l}{\textit{\textbf{Control and communication}}} \\
        Communication & \R No &
        \R VE.can and Bluetooth &
        Yes &
        \R Bluetooth and RS485 &
        \R Bluetooth \\
        Configurable & \R No & Yes & Yes & Yes & Yes \\
        GUI & No & Yes & Yes & Yes & Yes \\

        \midrule
        \multicolumn{6}{l}{\textit{\textbf{MPPT and charging}}} \\
        Prog. battery voltage &
        \R No &
        Yes &
        Yes &
        \R non-lithium batteries &
        Yes \\
        Tracking speed &
        15~Hz / 66.6(6)~ms &
        \R Fast (unknown) &
        Programmable &
        \R Unknown &
        \R Unknown \\
        Tracking efficiency &
        99\%+ typical &
        \R Unknown &
        Programmable &
        99\% &
        \R Unknown \\
        Electrical efficiency &
        96-99\% typical &
        98\% peak &
        \R Unknown &
        97\% &
        \R Unknown \\
        Charger profile &
        CC-CV &
        \R Unknown &
        Programmable &
        \R Unknown &
        CC-CV \\

        \midrule
        \multicolumn{6}{l}{\textit{\textbf{Hardware implementation}}} \\
        MCU & ATtiny461A-U & Unknown & STM32G474 & Unknown & Unknown \\
        MCU Details &
        8-bit, RISC, 64~MHz &
        -- &
        32-bit, Arm, 170~MHz  &
        -- &
        -- \\
        Transistor &
        FZT951  &
        Unknown &
        GS61008T  &
        Unknown &
        Unknown \\
        Transistor type & PNP BJT & Unknown & GaN FETs & Unknown & Unknown \\

        \midrule
        \multicolumn{6}{l}{\textit{\textbf{Economic}}} \\
        Price &
        \R 240~€ &
        \R 100-200~€ &
        \R Components: 250-280~€ \newline PCB: 20-100~€ \newline Total: 270-380~€ &
        \R 300-350~€ &
        \R 120-160~€ \\

        \bottomrule
        \end{tabularx}
        \end{adjustbox}
        \end{table}

    